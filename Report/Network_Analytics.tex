\chapter{Analisi della rete}
\section{Analisi del dataset}
Come affermato precedentemente, il dataset considerato mostra un'evidente distinzione in tre layer:
\begin{itemize}
	\item attacchi;
	\item messaggi;
	\item trade.
\end{itemize}
Ognuno di questi layer è stato analizzato per un periodo complessivo di 30 giorni, producendo di fatto un differente grafo per ogni giorno e per ogni layer (90 grafi totali).
Il dataset è reso disponibile in due differenti formati, \texttt{csv} e \texttt{graphml}; essendo stati riscontrati dei problemi di inconsistenza tra i due formati, e a seguito di indicazioni da parte del fornitore del dataset, abbiamo deciso di considerare unicamente il formato \texttt{graphml}.\\

La rete analizzata risulta quindi essere un multi-grafo diretto multi-layer. Considerato il fatto che la maggior parte delle misure di centralità di una rete sono pensate per grafi (e non per multi-grafi), è stato deciso di trasformare la rete in un grafo diretto pesato, dove il peso di un arco tra due nodi rappresentasse il numero di archi che connette i due nodi nel multi-grafo. Questa operazione è stata eseguita per ognuno dei 90 grafi, in modo da mantenere inalterati layer e \textit{timeslice}. Questa scelta ha permesso di ridurre drasticamente i tempi computazionali per le varie analisi che verranno presentate, preservando l'informazione grazie al peso degli archi.
È stata inoltre prodotta per ogni layer una versione aggregata sui 30 giorni, in modo da fornire una visone complessiva della rete e delle relazioni tra i vari nodi.
Tutti e tre i layer presentano archi orientati e con un \textit{timestamp}: esso è stato rimosso durante il passaggio a grafo pesato, ma è stato utilizzato per alcune analisi presentate in seguito.\\
Sono stati anche rilevati dei \textit{sefl-loop} nel layer dei messaggi: dato che Travian non offre la possibilità di inviare un messaggio a se stessi, abbiamo deciso di rimuovere questi archi, anche considerando il fatto che la loro cardinalità non era statisticamente rilevante ($0.03\%$).

Dall'analisi delle reti aggregate sui 30 giorni, riportata nella Tabella \ref{tab:summary}, notiamo come vi sia una predominanza di attività di attacco tra giocatori, seguita dai messaggi e dagli scambi. Il trand si conferma anche per quanto riguarda il numero di nodi attivi: troviamo un totale di nodi coinvolti in attività di combattimento superiore al numero di quelli che si scambiano messaggi o commerciano. Notiamo inoltre che unendo le tre reti in un unico grafo, otteniamo un numero di nodi superiore rispetto alle 3 reti separate. Ciò indica la presenza di giocatori che sono completamente estranei ad una di queste tre attività (\textit{e.g.}, soggetti che attaccano senza mai commerciare).\\
Il diametro della rete (valore rilevante esclusivamente per il layer dei messaggi) indica che il numero massimo di giocatori da cui deve passare l'informazione affinché venga trasmessa tra due qualunque nodi: nel nostro caso, questo valore è molto contenuto, pari solo a 9.\\
Il numero medio di vicini (\textit{i.g.}, giocatori in relazione diretta) ci mostra come, per quanto riguarda gli attacchi, vi sia la tendenza dei giocatori a selezionare bersagli preferenziali, concentrando gli attacchi verso quei villaggi che sono stati verificati già essere proficui. Al contrario, l'elevato valore per quanto riguarda gli scambi evidenzia come il mercato sia “globalizzato”, si tende a scambiare la propria merce con chiunque sia in grado di offrire in cambio un compenso ritenuto adeguato, senza la creazione di reti di scambio.\\
Il valore della reciprocità media esprime quanto nel grafo siano presenti collegamenti bidirezionali tra nodi. È apprezzabile come questo valore risulti essere estremamente elevato per quanto riguarda gli scambi (in quanto concepiti come vendita bidirezionale) e per i messaggi. Nel grafo degli attacchi troviamo un compattamento diametralmente opposto: solo pochi giocatori rispondono agli attacchi. Questo è probabilmente dovuto al fatto che i villaggi attaccati siano solitamente più deboli degli attaccanti, e non sarebbero quindi in grado di portare a termine con successo un attacco nei confronti degli assalitori.
\begin{table}[h]
	\centering
	\caption{Analisi macroscopica dei vari layer del dataset aggregati sui 30 giorni.}
	\begin{tabular}{|c|c|c|c|c|}
		\hline 
		Misura & Attacchi & Messaggi & Scambi & Complessivo \\ 
		\hline 
		$\#$Nodi & 4418 & 3092 & 2648 & 4612 \\ 
		\hline 
		$\#$Archi & 632391 & 380794 & 269483 & 1282668 \\ 
		\hline 
		Diametro & \textbf{/} & 9 & \textbf{/} & \textbf{/} \\ 
		\hline 
		Numero medio di vicini & 3.98 & 43.51 & 89.89 & / \\ 
		\hline 
		Reciprocità media & 0.039 & 0.704 & 0.938 & / \\ 
		\hline 
	\end{tabular} 
	\label{tab:summary}
\end{table}

\newpage
\subsection{Coefficiente di clustering medio}
Un'ulteriore misura analizzata è stata il coefficiente medio di \textit{clustering}, ovvero la misura della tendenza di un nodo a formare un \textit{Clique} (\textit{i.e.}, grafo completo). Questo comportamento è solito rilevarsi nelle reti sociali, dove i nodi tendono a formare delle community dense e coese rispetto al resto del grafo. Essendo questa una misura media, bisogna considerare il fatto che molti nodi ($\sim40\%$) non facciano parte di alcuna alleanza, e questo influenzi molto il valore medio ottenuto. 
I risultati ottenuti sono riportati in Figura \ref{fig:clustering}: vediamo come nel layer dei messaggi, che possiamo considerare una rete sociale, i valori medi sono  maggiori rispetto alla rete degli scambi, dove i vicini di un nodo non necessariamente sono soliti commerciare tra loro. In termini di assoluti, troviamo però valori molto bassi di questa misura. Questo è dovuto al fatto che nei 30 giorni, un giocatore contatti non solo i membri della sua alleanze, ma anche altri giocatori che non è detto che comunichino tra loro. Ci possiamo quindi aspettare che i nodi del vicinato di un generico giocatore siano più propensi a conoscersi e a comunicare tra loro, ma che i contatti con giocatori all'esterno della propria cerchia influenzi negativamente il valore di questa misura. Per quanto riguarda gli scambi, i nodi non sembrano essere più propensi a commerciare con nodi vicini più di quanto non lo siano con i nodi lontani, in quanto probabilmente l'unico parametro considerato in questo caso sia il profitto ottenuto dal \textit{trade}.
\begin{figure}
	\subfloat[Coefficiente di \textit{clustering} medio per i messaggi.]{%
		\includegraphics[width=.55\linewidth]{images/Aggregate/Clustering/messages}
	}
	\hfill
	\subfloat[Coefficiente di \textit{clustering} medio per i trade.]{%
		\includegraphics[width=.55\linewidth]{images/Aggregate/Clustering/trades}
	}
	\caption{Grafici delle distribuzioni dei coefficienti di \textit{clustering} per i layer dei messaggi e degli scambi.}
	\label{fig:clustering}
\end{figure}

\section{Analisi delle centralità}
\subsection{Analisi del grado}
\label{subsec:grado}
In Figura \ref{fig:degree} sono riportate le distribuzioni di \textit{in-degree}, \textit{out-degree} e grado complessivo dei vari layer della rete. Per \textit{out-degree} si intende il numero di azioni che un tale giocatore ha compiuto verso un altro giocatore, ovvero: averlo attacco, avergli mandato un messaggio oppure delle risorse. Analogamente, l'\textit{in-degree} è il numero di azioni che un giocatore ha “subito” da parte di altri: attacchi, messaggi o beni ricevuti. In termini computazionali, sono il numero di archi uscenti o entranti di un nodo. Dalle distribuzioni sono stati rimossi quei nodi considerati \textit{outlier}, ovvero quei nodi che si discostano oltre 3 volte dalla deviazione standard del \textit{fitting} dei dati con una distribuzione normale (per le distribuzioni di tali nodi, consultare l'Appendice A). I dati mostrano un andamento attendibile e ragionevole rispetto al dominio, con un elevato numero di nodi con grado contenuto e una percentuale minoritaria di nodi con alto grado.\\
\begin{figure}[p!]
	\begin{tabular}{ccc}
		\subfloat[Attacks in degree]{\includegraphics[width = 1.5in]{images/Aggregate/Degree/attacks_in_degree}} &
		\subfloat[Attacks out degree]{\includegraphics[width = 1.5in]{images/Aggregate/Degree/attacks_out_degree}} &
		\subfloat[Attacks total degree]{\includegraphics[width = 1.5in]{images/Aggregate/Degree/attacks_total_degree}}\\
		\subfloat[Messages in degree]{\includegraphics[width = 1.5in]{images/Aggregate/Degree/messages_in_degree}} &
		\subfloat[Messages out degree]{\includegraphics[width = 1.5in]{images/Aggregate/Degree/messages_out_degree}} &
		\subfloat[Messages total degree]{\includegraphics[width = 1.5in]{images/Aggregate/Degree/messages_total_degree}}\\
		\subfloat[Trades in degree]{\includegraphics[width = 1.5in]{images/Aggregate/Degree/trades_in_degree}} &
		\subfloat[Trades out degree]{\includegraphics[width = 1.5in]{images/Aggregate/Degree/trades_out_degree}} &
		\subfloat[Trades total degree]{\includegraphics[width = 1.5in]{images/Aggregate/Degree/trades_total_degree}}
	\end{tabular}
	\caption{In-degree, out-degree e grado complessivo dei vari layer della rete.}
	\label{fig:degree}
\end{figure}
 
Sono stati inoltre realizzati dei \textit{jointplot}, per mettere in relazione grado in ingresso ed in uscita dei nodi sui differenti layer, e i risultati sono riportati in Figura \ref{fig:join_degree}. Dai grafici emerge come, per quanto concerne le aggressioni, un elevato numero di attacchi compiuti corrisponde ad un basso numero di attacchi subiti, e viceversa. Questo dimostra la tendenza dei giocatori di attaccare avversari più deboli ripetutamente, al fine di ottenere tutte le risorse disponibili e privare questi villaggi della possibilità di espandersi. I giocatori più forti e più attivi non cercano il confronto ad armi pari, ma preferiscono attaccare villaggi che sono certi non risponderanno, in quanto militarmente più deboli.\\
Il grafico dei messaggi risulta essere più variegato, con tendenze a ricevere più messaggi di quelli inviati, ma non emerge un chiaro comportamento come nei due casi precedenti.\\
Per gli scambi, la situazione è diametralmente opposta, con un grafico che si distribuisce principalmente sulla diagonale, dove il numero di \textit{trade} in uscita è pari a quello in entrata. La maggior parte degli scambi è quindi a fine commerciali: tutti cercano di trarre vantaggio dallo scambio effettuato e solo pochi giocatori sono propensi ad effettuare donazioni.
\begin{figure}
	\centering
	\subfloat[\textit{Jointplot} attacchi.]{%
		\includegraphics[width=.55\linewidth]{images/Aggregate/Degree/jointplot/attacks_jointplot_no_outliers}
	}%
	\subfloat[\textit{Jointplot} messaggi.]{%
	\includegraphics[width=.55\linewidth]{images/Aggregate/Degree/jointplot/messages_jointplot_no_outliers}
	}\\
	\subfloat[\textit{Jointplot} trade.]{%
	\includegraphics[width=.55\linewidth]{images/Aggregate/Degree/jointplot/trades_jointplot}
	}%
	\caption{\textit{Jointplot} che pongono in relazione \textit{in-degree} e \textit{out-degree} del relativo layer.}
	\label{fig:join_degree}
\end{figure}

Infine, sono stati generati i \textit{jointplot} che pongono in relazione gli \textit{out-degree} di messaggi e attacchi o trade, rispettivamente in Figura \ref{subfig:joint_m_a} e \ref{subfig:joint_m_t}. La scelta delle misure da confrontare è ricaduta su messaggi e attacchi per capire se i giocatori militarmente molto attivi lo fossero anche sotto il profilo diplomatico, oppure fossero figure isolate. Similmente, sono stati confrontati messaggi e \textit{trade}, in modo da capire se i nodi fortemente attivi nel commercio, sia anche solito avere una fitta rete di comunicazione associata. Dal primo grafico si evince come numero di attacchi e numero di messaggi siano solitamente inversamente proporzionali, mentre non vi sembra essere una stretta relazione tra messaggi e scambi commerciali. Appare quindi confermata l'idea che i nodi militarmente più impegnati siano soliti avere associato un basso grado di comunicazione con altri giocatori, mentre i “commercianti” sono soliti essere fortemente connessi agli atri nodi da una rete di messaggi.
\begin{figure}
	\subfloat[\textit{Jointplot} tra messaggi e attacchi.\label{subfig:joint_m_a}]{%
		\includegraphics[width=.55\linewidth]{images/Aggregate/Degree/jointplot/messages_out-degree_vs_attacks_out-degree}
	}
	\hfill
	\subfloat[\textit{Jointplot} tra messaggi e scambi.\label{subfig:joint_m_t}]{%
		\includegraphics[width=.55\linewidth]{images/Aggregate/Degree/jointplot/messages_out-degree_vs_trades_out-degree}
	}
	\caption{\textit{Jointplot} che pongono in relazione l'\textit{out-degree} dei messaggi con quello degli attacchi o degli scambi.}
	\label{fig:join_mixed}
\end{figure}

\newpage
\subsection{Misure di betweenness e PageRank}
\label{subsec:centrality}
Per quanto riguarda le misure di centralità ulteriori rispetto ai gradi, abbiamo deciso di concentrare i nostri approfondimenti esclusivamente sui layer di messaggi e scambi, in quanto sulla rete di attacchi il calcolo di queste misure risulta essere poco sensato (non vi è alcun flusso informativo).\\
Le misure analizzate per i due layer considerati sono:
\begin{itemize}
	\item betweenness: misura che rappresenta la quantità di informazione passante per un dato nodo. Un elevto valore indica un nodo centrale nel flusso informativo/commerciale, con una grande capacità di collegamento tra porzioni della rete poiché in grado di fare da “ponte” tra vari insiemi di nodi poco connessi tra loro;
	\item PageRank: assegna ad ogni nodo un'importanza basandosi sul numero di archi entranti in quel nodo e sull'importanza dei nodi che si relazionano con esso: se nodi importanti hanno archi verso tale nodo, allora anch'esso è importante. Inoltre più un nodo è in relazione con altri nodi meno importanza “trasmetterà” ai singoli nodi.
\end{itemize}

Per il layer dei messaggi, i grafici delle due misure analizzate, riportati in Figura \ref{fig:messages_centrality}, mostrano una predominanza del contatto diretto rispetto al passaggio di informazione secondo una scala di gradi. Un distribuzione dei valori di betweenness concentrata in valori molto bassi, indica la presenza di molti collegamenti alternativi tra due nodi: non si presenta quindi la situazione di un passaggio “forzato” per un determinato nodo. Ciò è ragionevole, in quanto in Travian è possibile inviare un messaggio in modo diretto a qualunque nodo e, seppur esista una forma di gerarchia all'interno delle varie alleanze, ciò non impedisce la comunicazione tra tutti i livelli di essa. PageRank mostra un andamento simile, non individuando chiaramente dei nodi predominanti sugli altri. Le considerazioni per questa misura sono le medesime già fatte per i valori della betweenness. Si ricorda che dal dataset sono stati rimossi tutti i messaggi di \textit{broadcast}, ovvero le comunicazioni inviate dai gestori delle alleanze a tutti i membri di essa: questo ha certamente influenzato una compressione di valori delle due misure precedentemente analizzate.\\ 
\begin{figure}
	\subfloat[Centralità Betweenness.]{%
		\includegraphics[width=.55\linewidth]{images/Aggregate/Centrality/messages_betweenness}
	}
	\hfill
	\subfloat[Centralità Pagerank.]{%
		\includegraphics[width=.55\linewidth]{images/Aggregate/Centrality/messages_PageRank}
	}
	\caption{Grafici delle misure di centralità calcolate sul layer dei messaggi.}
	\label{fig:messages_centrality}
\end{figure}
Lo stesso comportamento è mostrato dall'analisi del grafo degli scambi (in Figura \ref{fig:trades_centrality}), con valori della betweenness ancora più compressi rispetto ai messaggi. Ciò è probabilmente dovuto all'assenza di figure di passaggio delle merci: non vi è un “mercante” che acquista beni da un insieme di giocatori per poi rivendere ad altri, ma i giocatori sono liberi di commerciare direttamente tra loro.
\begin{figure}
	\subfloat[\textit{Scatterplot} della misura di betweenness.]{%
		\includegraphics[width=.55\linewidth]{images/Aggregate/Centrality/trades_betweenness}
	}
	\hfill
	\subfloat[\textit{Scatterplot} della misura di Pagerank.]{%
		\includegraphics[width=.55\linewidth]{images/Aggregate/Centrality/trades_PageRank}
	}
	\caption{Grafici rappresentanti la distribuzione delle misure di centralità calcolate sul layer dei \textit{trade}.}
	\label{fig:trades_centrality}
\end{figure}

Abbiamo infine generato i \textit{jointplot} delle due misure analizzate, per individuare possibili correlazioni tra i valori delle due. I risultati, riportati in Figura \ref{fig:joint_centrality}, mostrano come la maggioranza dei nodi si concentri nei pressi dell'origine degli assi, a cui corrispondono bassi valori di entrambe le misure. Sono apprezzabili alcuni nodi un comportamento differente dalla moda in entrambi i grafici; in generale, i nodi che fungono da collegamento per le comunicazioni tendono a comunicare con nodi poco rilevanti. Viceversa, vi sono alcuni nodi che ricevono messaggi da nodi importanti, senza però connettere i nodi circostanti.
\begin{figure}
	\subfloat[\textit{Jointplot} che mette in relazione le misure di betweenness e PageRank per il layer dei messaggi.]{%
		\includegraphics[width=.55\linewidth]{images/Aggregate/Centrality/messages_jointplot}
	}
	\hfill
	\subfloat[\textit{Jointplot} che mette in relazione le misure di betweenness e PageRank per il layer dei \textit{trade}.]{%
		\includegraphics[width=.55\linewidth]{images/Aggregate/Centrality/trades_jointplot}
	}
	\caption{\textit{Jointplot} delle misure di centralità betweenness e PageRank calcolate rispettivamente sui messaggi e sugli scambi.}
	\label{fig:joint_centrality}
\end{figure}

\newpage 
\section{Analisi delle attività}
Un'ulteriore analisi condotta sul dataset è stata la distribuzione giornaliera di attività durante il periodo analizzato: ci siamo chiesti se fosse possibile delineare una qualche relazione tra i dati disponibili, in particolare tra numero di giocatori ed attività.
Il numero di giocatori attivi sul server in almeno uno dei tre layer è rappresentato in Figura \ref{fig:playereachday}, dove si può notare un drastico calo al giorno 5 (analisi più dettagliate saranno discusse in seguito).
\begin{figure}
	\centering
	\includegraphics[width=0.8\linewidth]{images/Activity/player_each_day}
	\caption{Numero di giocatori attivi in un dato giorno. Le linee rosse verticali rappresentano l'inizio di una nuova settimana.}
	\label{fig:playereachday}
\end{figure}
In Figura \ref{fig:activity_each_day}, dove troviamo rappresentato il numero di attività compiute in ognuno dei 30 giorni, emerge come complessivamente il numero di azioni eseguite dai giocatori decresca lungo il periodo analizzato, con picchi negativi durante i week-end e le festività. 
\begin{figure}
	\subfloat[Numero di archi presenti nei vari layer ad un dato giorno.]{%
		\includegraphics[width=.45\linewidth]{images/Activity/n_each_activity_each_day}
	}
	\hfill
	\subfloat[Somma del numero di archi presenti nei tre layer ad un dato giorno.]{%
		\includegraphics[width=.45\linewidth]{images/Activity/total_activities_each_day}
	}
	\caption{Le linee rosse verticali nel grafico di destra rappresentano l'inizio di una nuova settimana. Si può apprezzare un andamento complessivamente decrescente, soprattutto in corrispondenza del week-end e di festività.}
	\label{fig:activity_each_day}
\end{figure}
In Figura \ref{fig:activity_players} è possibile osservare la relazione tra numero di giocatori e attività eseguite. Come ci aspettavamo, l'andamento delle due curve risulta essere appaiato, e mantiene il trend decrescente individuato dai grafici in Figura \ref{fig:activity_each_day} e \ref{fig:playereachday}.
\begin{figure}
	\subfloat[Numero di giocatori che hanno effettuato almeno un'azione nel layer specificato ad un dato giorno.]{%
		\includegraphics[width=.45\linewidth]{images/Activity/player_each_activity_each_day}
	}
	\hfill
	\subfloat[Andamento nei 30 giorni del numero di giocatori e delle attività totali. Si noti che per l'asse delle ordinate è stato utilizzato una scala logaritmica.]{%
		\includegraphics[width=.45\linewidth]{images/Activity/players_vs_activities_each_day}
	}
	\caption{L'andamento conferma il trend decrescente evidenziato anche in precedenza.}
	\label{fig:activity_players}
\end{figure}
La brusca decrescita rilevabile sia nel grafico delle attività, sia nel numero di giocatori attivi tra il giorno 3 ed il giorno 5 sarà descritta successivamente, nella Sezione \ref{subsec:eot}.

In ultima battuta, abbiamo analizzato quali fossero gli orari dove si concentrasse il maggior numero di attività: in Figura \ref{fig:activity_layers} sono riportati i risultati ottenuti (si consideri il fatto che, con il valore 18 sono intese tutte le azioni compiute tra le 18:00 e le 18:59). Notiamo come la maggioranza delle azioni sia eseguita tra le ore 14 e le 22, con un picco introno alle ore 18. L'andamento degli attacchi risulta essere più “liscio”, probabilmente per l'elevata attività mattutina dovuta al tentativo di appropriarsi delle risorse accumulate nella notte dagli altri villaggi. L'andamento crescente rilevato nella fascia oraria sopracitata è probabilmente dovuto alla concomitanza con la fine degli orari scolastici e lavorativi; i giocatori, avendo a disposizione più tempo libero, intensificano le attività online.\\
\begin{figure}
	\subfloat[Numero di attività per layer effettuate nei 30 giorni in una specifica fascia oraria.]{%
		\includegraphics[width=.45\linewidth]{images/Activity/different_over_hours}
	}
	\hfill
	\subfloat[Numero di attività complessive per fascia oraria.]{%
		\includegraphics[width=.45\linewidth]{images/Activity/total_over_hours}
	}
	\caption{Attività sul server divisa per fascia oraria. Andamento crescente dalle ore 6 alle 18, poi decrescente fino alle 23}
	\label{fig:activity_layers}
\end{figure}


\section{Ricerca di comportamenti sospetti}
Travian sostiene di aver implementato nel suo strumento di chat un metodo per impedire lo spam di messaggi verso altri utenti: se viene rilevato un grande numero di messaggi in uscita in un intervallo di tempo ristretto, il sistema blocca l'accesso del giocatore alla chat per periodi crescenti, fino ad un ban permanente.\\
Abbiamo voluto verificare l'efficacia di questo meccanismo, cercando quei nodi con un grande numero di messaggi in uscita a cui non corrispondessero un equo numero di messaggi in ingresso.
Non sono stati considerati possibili spammer nodi appartenenti ad alleanze con più di 10 membri, in quanto i moderatori della comunità sono soliti eliminare utenti con un'attività di questo tipo, considerandoli dannosi per la crescita della stessa.
La nostra analisi ha confermato l'assenza di figure definibili spammer, ovvero con un rapporto messaggi inviati/messaggi ricevuti superiore a $10\times$, e con un numero minimo di messaggi in uscita pari a 100.

\section{Ricerca di \textit{side-villages}}
Una pratica comune, seppur vietata dal regolamento di Travian, è quella del cosiddetto \textit{side-village} ovvero un secondo profilo, parallelo al principale, sfruttato per la produzione di risorse che vengono poi inviate al profilo principale sotto forma di donazione. Tale meccanismo è promosso dalla politica di Travian di fornire ai nuovi villaggi uno scudo per gli attacchi di 30 giorni, terminato il quale è possibile effettuare commerci e donazioni.\\
Abbiamo basato la rilevazione di questo comportamento sul rapporto tra \textit{trade} in uscita e \textit{trade} in ingresso: un valore superiore a $5\times$ ci ha fatto ritenere il comportamento come sospetto, e degno di più attente analisi da parte degli amministratori.\\
Dall'analisi sono emersi un totale di 10 account sospetti, in particolare i giocatori con ID: 1439, 12362, 9444, 11407, 3004, 10785, 12069, 10029, 7568, 1346.\\
Abbiamo altresì notato come a questi nodi sia associato un bassissimo grado di messaggi in entrata ed in uscita, accrescendo ulteriormente il nostro sospetto della presenza di un comportamento illecito.

\section{Studio delle alleanze}
Il dataset presenta, in aggiunta ai file \texttt{graphml}, un file di testo contenente, giorno per giorno, tutte le alleanze registrate, con i relativi membri.\\
Al fine di effettuare analisi più approfondite su queste community, si è delineata la necessità di identificare le alleanze nel tempo e tenere traccia dei relativi membri.
La maggiore difficoltà riscontrata in questa operazione si è rivelata essere l'assegnazione di un nome univoco a queste community; i file di testo forniti contengono, infatti, esclusivamente gli identificatori dei nodi appartenenti ad una community, senza però stabilire a quale alleanza appartengano.
Questo è risultato essere un problema passando da un giorno all'altro, in quanto le alleanze non sono rappresentate nel medesimo ordine, e alcune si potrebbero sciogliere o creare nel frattempo.

Le informazioni rilevanti che abbiamo estrapolato sono due:
\begin{itemize}
	\item per ogni nodo, il nome delle alleanze di cui ha fatto parte in ogni giorno;
	\item per ogni alleanza: i relativi membri in ogni giorno. 
\end{itemize}
Per ottenere ciò, è stato implementato un algoritmo \textit{ad hoc}; tale algoritmo si divide principalmente in due fasi:
\begin{enumerate}
	\item analisi delle community al primo giorno;
	\item analisi dell'evoluzione delle community.
\end{enumerate}
Per quanto concerne il primo giorno, ogni riga del file è stato considerato un'alleanza: a tale alleanza è stato assegnato un nome caratterizzato da un identificatore incrementale, in modo da garantire l'unicità dei nomi. 
È stato quindi sufficiente leggere il file riga per riga, aggiornando le relative strutture dati (quelle contenenti le informazioni sopracitate).
Una volta stabilite le alleanze del primo giorno, il problema presentatosi per i giorni successivi è stato quello di stabilire se un'alleanza si fosse disgregata oppure solo alcuni membri al suo interno fossero cambiati.
Abbiamo deciso di affrontare il problema in maniera \textit{naive}, riducendo il problema a stabilire il nome delle alleanze giorno per giorno; per far ciò si è adottato un approccio di votazione dei membri di ogni alleanza.
Per ogni file, viene letta riga per riga il contenuto e viene fatto votare ogni membro rispetto al nome che vorrebbe dare all'alleanza: ogni giocatore vorrà votare con il nome della community di cui faceva parte al giorno precedente, se ne faceva parte (assumiamo che ogni giocatore voglia rimanere nella stessa alleanza), altrimenti non vota.
Tale approccio ci garantisce che le alleanze in cui sono cambiati pochi membri mantengano lo stesso nome, mentre quelle che hanno subito cambiamenti significativi nella composizione tenderanno a smettere di esistere.
Questa soluzione presenta un problema nel momento in cui due alleanza siano candidate ad avere lo stesso nome, evento possibile nel caso una community si spezzi in due sotto-community di pari cardinalità, oppure semplicemente quando più alleanze hanno al loro interno un numero maggiore di giocatori provenienti dalla stessa alleanza del giorno precedente.
Per risolvere tale problema, si è deciso di utilizzare una politica \textit{lazy} sull'assegnamento del nome: si sceglie di assegnare lo stesso nome alla community avente la cardinalità dell'intersezione tra l'alleanza al giorno precedente con l'insieme di membri al giorno di interesse massima; se due intersezioni hanno la stessa cardinalità massima, si sceglie casualmente la community tra quelle candidate.
Tutte le community che non hanno “vinto” nella risoluzione di tali conflitti, vedranno assegnato loro un nuovo nome, sempre mediante un \textit{id} univoco incrementale.
Una volta risolta l'assegnazione dei nomi, le struttre dati vengono aggiornate di conseguenza.

È stata così prodotta una rappresentazione di ogni singola community in ognuno dei 30 giorni contenente un identificatore univoco per l'alleanza e tutti i suoi membri in ogni istante temporale. Il numero complessivo di community così identificate nei 30 giorni è stato pari a 323: dato un così alto numero di community registrate, abbiamo deciso di considerare rilevanti ai fini degli studi successivi esclusivamente quelle alleanze che vantassero un numero di giocatori pari almeno a 10.

\subsection{Studio della densità e della reciprocità delle community}
\label{sec:density}
Un parametro indice della coesione sociale presente tra i membri di una comunità è quello della densità, ovvero il numero totale di interazioni instaurate tra i nodi rispetto al massimo valore di interazioni possibili. Associato a questa misura troviamo spesso la reciprocità, vale a dire il numero di legami bidirezionali rispetto al numero di legami totali. \\
Le due misure sono state calcolate sui layer di messaggi e scambi, in modo da verificare quanto stabile fosse la rete di comunicazione e commerci all'interno delle alleanze più rilevanti.
I risultati, riportati in Figura \ref{fig:density_reciprocity}, mostrano come il valore di reciprocità di messaggi e \textit{trade} sia molto inferiore a quelli mostrati dall'analisi dei dati aggregati, riportati in Tabella \ref{tab:summary}. Si tenga però a mente che il grafo analizzato in quella tabella è il risultato della somma dei 30 giorni, mentre l'analisi qui condotta mostra il valore della rete per ogni giorno, senza considerare gli archi precedenti o successivi.\\
I bassi valori di densità per quanto riguarda i \textit{trade} all'interno della rete evidenziano una tendenza dei membri di una community a commerciare con giocatori esterni, probabilmente per evitare di farsi concorrenza a vicenda. Il valore relativamente basso della reciprocità per questo layer si concretizza in un elevato numero di donazioni, pratica molto diffusa all'interno delle alleanze e che garantisce bonus ai giocatori che superano una certa soglia di beni donati. Inoltre, in caso di attacchi ripetuti, i giocatori sono soliti inviare ai membri dell'alleanza le proprie risorse, in modo da risultare bersagli meno appetibili.\\
Per quanto riguarda i messaggi, conosciamo dai pubblicatori del dataset che non sono presenti i messaggi di \textit{broadcast} all'interno del dataset.  Ciononostante, la densità media del layer dei messaggi è approssimativamente 10$\times$ superiore rispetto all'intera rete, mostrando una più fitta rete di comunicazione tra i vari membri della community.\\
\begin{figure}
	\subfloat[Densità di messaggi e scambi delle alleanze più rilevanti.]{%
		\includegraphics[width=.45\linewidth]{images/Community/Density_Reciprocity/msg_vs_trades_density}
	}
	\hfill
	\subfloat[Reciprocità di messaggi e scambi delle alleanze più rilevanti.]{%
		\includegraphics[width=.45\linewidth]{images/Community/Density_Reciprocity/msg_vs_trades_reciprocity}
	}
	\caption{Andamento di densità e reciprocità delle alleanze più rilevanti.}
	\label{fig:density_reciprocity}
\end{figure}

Un'ulteriore analisi compiuta sulle community è il numero di esse che siano risultati assenti (ovvero con densità pari a 0) in uno dei due layer analizzati. Il grafico ottenuto, riportato in Figura \ref{fig:densityzeroper}, mostra come quasi una community su quattro sia inattiva a livello degli scambi, mentre i membri circa uno su dieci non comunicano tra loro all'interno di una giornata. I picchi di inattività, inoltre, corrispondono ad apici negativi nei valori di densità e reciprocità nei grafici in Figura \ref{fig:density_reciprocity}. Ciò è perfettamente ragionevole, in quanto una community con densità zero influisce negativamente sul valore medio della misure per le alleanze.
\begin{figure}
	\centering
	\includegraphics[width=0.9\linewidth]{images/Community/Density_Reciprocity/density_zero_per}
	\caption{Rapporto percentuale tra numero di community rilevanti con densità pari a 0 rispetto al numero totale di community rilevanti. In blu troviamo il layer dei messaggi, mentre i trade sono rappresentati in verde.}
	\label{fig:densityzeroper}
\end{figure}


\subsection{Analisi della community principale}
\label{subsec:comm_princ}
Per determinare quale fosse l'alleanza principale (la più rilevante) contenuta nel dataset, abbiamo deciso di utilizzare come metrica la dimensione e la stabilità della community sui 30 giorni. Abbiamo così scelto l'alleanza di cardinalità massima e con minore deviazione standard rispetto alla sua dimensione, che si è rivelata essere quella con identificatore 43. Le analisi realizzate in seguito sono replicabili per ogni alleanza presente nel server: per questioni di tempo, in questo lavoro è stata effettuata solo sulla comunità principale. Riportiamo inoltre che per le prime 4 community in ordine di grandezza, la dimensione media e la relativa deviazione standard non mostrano differenze significative rispetto alla community scelta. Possiamo quindi immaginare che all'interno del server vi sia un piccolo sottoinsieme di alleanze principali, considerabili molto numerose e stabili, affiancate ad un ben più alto numero di community meno rilevanti e molto più dinamiche, che crescono di dimensione, si scindono o addirittura scompaiono.\\
Per l'analisi della community è stato analizzato il layer delle comunicazioni, in quanto abbiamo ritenuto che fosse quello maggiormente significativo per estrarre analisi interessanti dal punto di vista sociale e del flusso informativo. Abbiamo infatti rilevato che gli scambi sono eseguiti in maniera indiscriminata con tutti i giocatori del server, mentre non è ragionevole analizzare il layer degli attacchi all'interno della stessa community.
Sono state ripetute le misure di centralità trattate nelle Sezioni \ref{subsec:grado} e \ref{subsec:centrality}, con l'aggiunta del calcolo della \textit{closeness}, ovvero una misura di quanto un nodo sia vicino al flusso informativo presente nella rete. I risultati sono riportati nelle Figure \ref{fig:comm_mean}  e \ref{fig:comm_nodes}, dove sono riportate le distribuzioni delle medie delle misure e i valori per ogni nodo nei 30 giorni. Dalle distribuzioni riportare risulta evidente come vi sia una grande maggioranza di membri all'interno dell'alleanza che risultano essere poco influenti, mentre ad una parte minoritaria dei nodi sono associati valori più elevati.\\
Un'analisi approfondita mostra come i valori di centralità sopra la media siano associati in modo ricorrente agli stessi nodi: essi corrispondono probabilmente alle figure più rilevanti all'interno dell'alleanza, quali il leader, i suoi diretti sottoposti e figure come il reclutatore.
Emerge quindi una gestione piramidale della community, con pochi membri ai vertici ed una moltitudine di componenti poco rilevanti; questo è ragionevole in un gioco basato sul predominio militare, in cui vige la legge del più forte. Un'organizzazione piramidale permette ai pochi giocatori più potente di prendere le decisioni e imporle sui membri dell'alleanza meno importanti. importanza 
\begin{figure}
	\begin{tabular}{cc}
		\subfloat[Distribuzione dell'\textit{out-degree} medio sui 30 giorni]{\includegraphics[width = .5\textwidth]{images/Most_important_community/distribution/out-degree_distribution}} &
		\subfloat[Distribuzione della betweenness media sui 30 giorni.]{\includegraphics[width = .5\textwidth]{images/Most_important_community/distribution/betweenness_distribution}}\\
		\subfloat[Distribuzione della closeness sui 30 giorni.]{\includegraphics[width = .5\textwidth]{images/Most_important_community/distribution/closeness_distribution}} &
		\subfloat[Distribuzione media del valore di PageRank sui 30 giorni.]{\includegraphics[width = .5\textwidth]{images/Most_important_community/distribution/pagerank_distribution}}
	\end{tabular}
	\caption{In figura sono riportate le distribuzioni medie delle misure medie di \textit{out-degree}, betweenness, closeness e PageRank sui 30 giorni.}
	\label{fig:comm_mean}
\end{figure}
\begin{figure}
	\begin{tabular}{cc}
		\subfloat[Distribuzione di \textit{out-degree} di ogni nodo sui 30 giorni.]{\includegraphics[width = .5\textwidth]{images/Most_important_community/each_node/out-degree_each_node}} &
		\subfloat[Distribuzione di betweenness di ogni nodo sui 30 giorni.]{\includegraphics[width = .5\textwidth]{images/Most_important_community/each_node/betweenness_each_node}}\\
		\subfloat[Distribuzione di closeness di ogni nodo sui 30 giorni.]{\includegraphics[width = .5\textwidth]{images/Most_important_community/each_node/closeness_each_node}} &
		\subfloat[Distribuzione di PageRank di ogni nodo sui 30 giorni.]{\includegraphics[width = .5\textwidth]{images/Most_important_community/each_node/pagerank_each_node}}
	\end{tabular}
	\caption{In fiura sono riportate le distribuzioni delle misure \textit{out-degree}, betweenness, closeness e PageRank di ogni nodo dell'alleanza principale per ognuno dei 30 giorni.}
	\label{fig:comm_nodes}
\end{figure}

Abbiamo poi analizzato i valori di densità e di reciprocità di questa community e li abbiamo confrontati con i valori medi ottenuti nella Sezione \ref{sec:density}, cercando di stabilire se vi fossero differenze rilevanti. Le Figure \ref{fig:most_imp_mess} e \ref{fig:most_imp_trade} mostrano valori di densità e reciprocità dei \textit{trade} nettamente maggiori all'interno di questa alleanza rispetto alla media delle altre community. Lo stesso vale per la densità relativa ai messaggi, mentre per quanto riguarda la reciprocità troviamo valori lievemente inferiori rispetto alla media, con picchi di positivi in alcuni giorni. Questo tipo di dato può essere causato, come già affermato precedentemente, dalla mancanza nel dataset dei messaggi di \textit{broadcast}, utilizzati per le comunicazioni da parte dei membri di alto grado dell'alleanza. L'immagine che emerge da questa analisi è quella di una community più coesa della media, con una discreta attività di scambi e commerci interni alla community che va oltre la media delle altre alleanze.
\begin{figure}
	\subfloat[Confronto tra densità di messaggi della community analizzata rispetto alla media.]{%
		\includegraphics[width=.45\linewidth]{images/Most_important_community/density/msg_vs_most_density}
	}
	\hfill
	\subfloat[Confronto tra reciprocità dei messaggi della community analizzata rispetto alla media.]{%
		\includegraphics[width=.45\linewidth]{images/Most_important_community/density/msg_vs_most_reciprocity}
	}
	\caption{Andamento di densità e reciprocità dei messaggi della community a confronto con le medie della rete.}
	\label{fig:most_imp_mess}
\end{figure}
\begin{figure}
	\subfloat[Confronto tra densità di trade della community analizzata rispetto alla media.]{%
		\includegraphics[width=.45\linewidth]{images/Most_important_community/density/trades_vs_most_density}
	}
	\hfill
	\subfloat[Confronto tra reciprocità dei trade della community analizzata rispetto alla media.]{%
		\includegraphics[width=.45\linewidth]{images/Most_important_community/density/trades_vs_most_reciprocity}
	}
	\caption{Andamento di densità e reciprocità dei trade della community a confronto con i valori medi delle community.}
	\label{fig:most_imp_trade}
\end{figure}

Ci siamo poi chiesti se i membri all'interno della community coordinassero i propri attacchi verso bersagli specifici, oppure vi fosse la tendenza ad aggredire bersagli differenti. Abbiamo quindi isolato in ogni giorno i nodi della community ed i relativi bersagli, conteggiando poi quanti differenti membri dell'alleanza avessero attaccato il medesimo nodo. I dati così ricavati hanno fatto emergere la propensione dei membri a non aggredire il medesimo nodo, ma di spartirsi i bersagli, probabilmente al fine di massimizzare il bottino ottenuto.
Seppur il massimo numero di differenti membri che hanno attaccato il medesimo nodo si è rivelato essere contenuto (pari al massimo a 7), differente è il numero di attacchi portati dal singolo giocatore al medesimo nodo. I dato ottenuti hanno infatti mostrato come l'attacco non si limiti al singolo evento, ma sia ripetuto all'interno della giornata a più riprese, anche oltre le 15 volte.
Crediamo che questo tipo di comportamento possa essere dovuto all'individuazione di un bersaglio particolarmente ricco di bottino e con scarse difese, una facile preda per l'attaccante, oppure a meccanismi di controllo e dominio di alcuni villaggi, che vengono sistematicamente attaccati al fine di non permetterne lo sviluppo.

\newpage
\subsection{Evoluzione nel tempo}
\label{subsec:eot}
Come affermato nella Sottosezione \ref{subsec:comm_princ}, il server è costituito da un grande numero di alleanze registrate, ma quante di queste sono davvero da considerarsi come influenti? Abbiamo quindi indagato come il numero di alleanze registrate variasse nel tempo, cercando di identificare se vi fosse un chiaro trand di accorpamento delle alleanze o di disaggregazione di esse. I risultati ottenuti, mostrati in Figura \ref{fig:allianceseachday}, mostrano una progressiva diminuzione sia del numero di alleanze totali, sia del numero di alleanze considerabili rilevanti (si ricorda che come metrica di rilevanza è stata utilizzato il numero minimo di membri pari a 10). La percentuale di community rilevanti sul totale si mantiene pressochè stabile, oscillando tra il $25\%$ e il $35\%$.\\
L'andamento decrescente mostrato in Figura \ref{fig:allianceseachday} sembra confermare il trand presente anche nelle Figure \ref{fig:activity_each_day} e \ref{fig:playereachday}, che mostra un progressivo spopolamento del server, con conseguente diminuzione del numero di attività totali eseguite ogni giorno.

Data la presenza ricorrente in questi grafici di un picco negativo al giorno 5, abbiamo deciso di indagare più approfonditamente questo fenomeno, sospettando che un qualche intervento esterno potesse aver alterato il numero di giocatori presenti. Analizzando gli insiemi dei giocatori presenti prima e dopo il quinto giorno, abbiamo rilevato che ben 715 giocatori che erano presenti in almeno uno dei layer della rete nei gironi precedenti al crollo, non hanno compiuto alcun tipo di attività nei successivi 25 giorni. Questo tipo di riscontro sembrerebbe essere compatibile con un evento esterno che abbia rimosso i giocatori (\textit{e.g.}, ban) o che li abbia portati ad andarsene (\textit{e.g.}, apertura di un nuovo server). Sfortunatamente non siamo riusciti a trovare conferme di queste ipotesi, in quanto non siamo riusciti a trovare alcuna prova online che confermasse o smentisse la nostra tesi.
\begin{figure}
	\centering
	\includegraphics[width=0.85\linewidth]{images/alliances_each_day}
	\caption{Andamento del numero di alleanze registrate nel tempo. Nei tre livelli sono riportati rispettivamente: Numero di alleanze per giorni, numero di alleanze rilevanti per giorno e percentuale di alleanze rilevanti sul totale per giorno.}
	\label{fig:allianceseachday}
\end{figure}

\newpage
\subsection{Interazioni tra le alleanze}
Al fine di visualizzare i meccanismi di interazione tra le principali alleanze, sono state prodotte delle rappresentazioni dei sottografi contenenti i nodi delle 4 community più grandi e delle relative interazioni. Per produrre i sottografi di ogni layer è stato utilizzato il multi-grafo ottenuto dall'aggregazione dei 30 giorni di attività; il risultato è stato poi trasformato in grafo pesato, in modo da ridurre il numero di archi e far emergere solo i contatti tra i giocatori.\\
Dal grafo degli attacchi, rappresentato in Figura \ref{suba:inter}, emerge come la community più aggressiva (e la più aggredita a sua volta) sia quella denotata dal colore grigio (alleanza con \textit{id} 44). Il numero di attacchi è complessivamente molto contenuto, e sono rarissimi casi di attacchi tra membri della stessa comunità. La tendenza che emerge è quindi quella di non aggressione tra le alleanze più grandi, che tendono a concentrare i propri attacchi su villaggi isolati o facenti parti di comunità dalla ridotta dimensione. Ciò sembra ragionevole, in quanto durante la fase di \textit{mid-game} non è conveniente aprire guerre o combattere contro altre alleanze forti. Conviene piuttosto rinforzarsi, attaccando villaggi più deboli e stabilendo di fatto uno \textit{status quo} che porterà le principali alleanze a scontrarsi solo nel late game.

La rappresentazione del grafo degli scambi, riportata in Figura \ref{subb:inter}, conferma l'idea di un mercato “globale”, con la formazione di una fitta rete di scambi anche tra membri di comunità diverse.
Per quanti riguarda il layer dei messaggi, è stata condotta un'analisi sotto due punti di vista differenti. In primo luogo sono state rappresentate tutte le interazioni avvenute all'interno e all'esterno delle community, producendo la rappresentazione in Figura \ref{subc:inter}. Da questa rappresentazione si nota come le comunicazioni \textit{intra-community} siano fortemente predominanti rispetto a quelle \textit{inter-community}; caso a parte sono però le due alleanze rosse e blu, che sembrano essere fuse e interconnesse da una fitta rete di comunicazione. Notiamo inoltre la presenza di alcuni giocatori (rappresentati in giallo) che hanno effettuato un cambio di community nei 30 giorni, passando da quella blu a quella rossa. L'ipotesi avanzata in questo caso è una delle due alleanze (la blu in questo caso) sia una sorta di accademia, un'alleanza satellite della comunità rossa nella quali i giocatori vengono messi alla prova e da cui, nel caso si liberi un posto, i giocatori siano spostati nell'alleanza madre. Questo meccanismo, non essendo vietato, si è molto diffuso in giochi come Travian e permette alla community principale di rimanere stabile e ben protetta, formando di fatto una macro-alleanza con community satellite.\\
Indagando poi sui contratti \textit{inter-comunità}, abbiamo prodotto il grafo in Figura \ref{subd:inter}, dove sono rappresentati solo i contatti tra giocatori di comunità diverse. Escludendo la forte connessione tra i giocatori rossi e quelli blu, vediamo come il numero di nodi di altri colori sia estremamente limitato: ciò indica che i contatti tra due community avvengono solo attraverso poche figure prestabilite. Analizzando il comportamento di questi nodi nel tempo, è possibile intuire se essi svolgano la funzione di reclutatori, ambasciatori o se possano essere spie che passano informazioni alle comunità rivali. Poiché tali analisi richiederebbero un cospicuo tempo sia in termini computazionali che non, non sono stati effettuate in questo lavoro ma verranno proposti come sviluppi futuri.
\begin{figure}
	\subfloat[Aggresioni tra i giocatori delle 4 più grandi community.\label{suba:inter}]{%
		\includegraphics[width=.45\linewidth]{images/Interaction/attacks}
	}
	\hfill
	\subfloat[Scambi tra i giocatori delle 4 più grandi community.\label{subb:inter}]{%
		\includegraphics[width=.45\linewidth]{images/Interaction/trades}
	}
	\caption{Attacchi e scambi tra i giocatori delle 4 più grandi alleanze.}
	\label{fig:inter_att_tra}
\end{figure}

\begin{figure}
	\subfloat[Messaggi tra i giocatori delle 4 più grandi community.\label{subc:inter}]{%
		\includegraphics[width=.45\linewidth]{images/Interaction/messages}
	}
	\hfill
	\subfloat[Contatti tra i giocatori di community differenti rispetto alle 4 più grandi community.\label{subd:inter}]{%
		\includegraphics[width=.45\linewidth]{images/Interaction/communities_contact}
	}
	\caption{Messaggi \textit{inter-intra} community relativamente alle 4 più grandi alleanze.}
	\label{fig:inter_messages}
\end{figure}
